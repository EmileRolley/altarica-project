\documentclass[a4paper]{book}
\usepackage{fullpage}

\usepackage[utf8]{inputenc}
\usepackage[T1]{fontenc}
\usepackage[francais]{babel}

\usepackage{latexsym}
\usepackage{fancyhdr}
\usepackage{makeidx}
\usepackage{graphics}
\usepackage{graphicx}
\usepackage{longtable}
\usepackage{moreverb}
\usepackage{listings}

\newcommand{\altarica}{{\sc AltaRica}}

\begin{document}

\title{Master 1, Conceptions Formelles\\
Projet du module \altarica\\
Synthèse (assistée) d'un contrôleur du niveau d'une cuve}

\date{}

\author{Nom1 \and Nom2 \and Nom3}

\maketitle

\chapter{Le sujet}
\section{Cahier des charges}

Le système que l'on souhaite concevoir est composé~:
\begin{itemize}
\item d'un réservoir contenant {\bf toujours} suffisamment d'eau pour alimenter l'exploitation,
\item d'une cuve,
\item de deux canalisations parfaites amont reliant le réservoir à la cuve, et permettant d'amener l'eau à la cuve,
\item d'une canalisation parfaite aval permettant de vider l'eau de la cuve,
\item chaque canalisation est équipée d'une vanne commandable, afin de réguler l'alimentation et la vidange de la cuve,
\item d'un contrôleur.
\end{itemize}

\subsection{Détails techniques}

\subsubsection{La vanne}
Les vannes sont toutes de même type, elles possèdent trois niveaux de débits correspondant à trois diamètres d'ouverture~: 0 correspond à la vanne fermée, 1 au diamètre intermédiaire et 2 à la vanne complètement ouverte. Les vannes sont commandables par les deux instructions {\tt inc} et {\tt dec} qui respectivement augmente et diminue l'ouverture. Malheureusement, la vanne est sujet à défaillance sur sollicitation, auquel cas le système de commande devient inopérant, la vanne est désormais pour toujours avec la même ouverture.

\subsubsection{La Cuve}
Elle est munie de $nbSensors$ capteurs (au moins quatre) situés à $nbSensors$ hauteurs qui permettent de délimiter $nbSensors+1$ zones. La zone 0 est comprise entre le niveau 0 et le niveau du capteur le plus bas; la zone 1 est comprise entre ce premier capteur et le second, et ainsi de suite.

Elle possède en amont un orifice pour la remplir limité à un débit de 4, et en aval un orifice pour la vider limité à un débit de 2.  

\subsubsection{Le contrôleur}
Il commande les vannes avec les objectifs suivants ordonnés par importance~:
\begin{enumerate}
\item Le système ne doit pas se bloquer, et le niveau de la cuve ne doit jamais atteindre les zones 0 ou $nbSensors$.
\item Le débit de la vanne aval doit être le plus important possible.
\end{enumerate}

On fera également l'hypothèse que les commandes ne prennent pas de temps, et qu'entre deux pannes et/ou cycle {\em temporel}, le contrôleur à toujours le temps de donner au moins un ordre. Réciproquement, on fera l'hypothèse que le système à toujours le temps de réagir entre deux commandes.

\subsubsection{Les débits}
Les règles suivantes résument l'évolution du niveau de l'eau dans la cuve~:
\begin{itemize}
\item Si $(amont > aval)$ alors au temps suivant, le niveau aura augmenté d'une unité.
\item Si $(amont < aval)$ alors au temps suivant, le niveau aura baissé d'une unité.
\item Si $(amont = aval = 0)$ alors au temps suivant, le niveau n'aura pas changé.
\item Si $(amont = aval > 0)$ alors au temps suivant, le niveau pourra~:
  \begin{itemize}
  \item avoir augmenté d'une unité,
  \item avoir baissé d'une unité,
  \item être resté le même.
  \end{itemize}
\end{itemize}

\section{L'étude}

\subsection{Rappel méthodologique}
Comme indiqué en cours, le calcul par point fixe du contrôleur est exact, mais l'opération de projection effectuée ensuite peut perdre de l'information et générer un contrôleur qui n'est pas satisfaisant. Plus précisemment, le contrôleur \altarica\ généré~:
\begin{itemize}
\item ne garanti pas la non accessibilité des \emph{Situations Redoutées}.
\item ne garanti pas l'absence de \emph{nouvelles situations de blocages}.
\end{itemize}

Dans le cas ou il existe toujours \emph{des situations de blocages ou redoutées}, vous pouvez au choix~:
\begin{enumerate}
\item Corriger manuellement le contrôleur calculé (sans doute très difficile).
\item Itérer le processus du calcul du contrôleur jusqu'à stabilisation du résultat obtenu. 
  \begin{itemize}
  \item Si le contrôleur obtenu est sans blocage et sans situation redoutée, il est alors correct.
  \item Si le contrôleur obtenu contient toujours des blocages ou des situations redoutées, c'est que le contrôleur initial n'est pas assez performant, mais rien ne garanti que l'on soit capable de fournir ce premier contrôleur suffisemment performant.
  \end{itemize}
\end{enumerate}

{\bf Remarque} : Pour vos calculs, vous pouvez utiliser au choix les commandes~:
\begin{itemize}
\item {\tt altarica-studio xxx.alt xxx.spe}
\item {\tt arc -b xxx.alt xxx.spe}
\item {\tt make} pour utiliser le fichier GNUmakefile fourni.
\end{itemize}

\subsection{Le travail a réaliser}
L'étude consiste à étudier le système suivant trois paramètres~:
\begin{enumerate}
\item $nbFailures$~: une constante qui est une borne pour le nombre de vannes pouvant tomber en panne.
\item Le contrôleur initial qui peut être soit {\tt Ctrl}, soit {\tt CtrlVV}.
\item Une éventuelle optimisation pour améliorer le débit aval.
\end{enumerate}

Les questions auxquelles vous devez répondre sont dans le fichier {\tt fichier rapport.tex}. Elles correspondent aux interrogations suivantes~:
\begin{enumerate}
\item Est-il possible de contrôler en évitant les blocages et les situations critiques ?
\item Si oui, donnez quelques caractéristiques de ce contrôleur, si non, expliquez pourquoi.
\item Est-il possible de contrôler en optimisant le débit aval et en évitant les blocages et les situations critiques ?
\item Si oui, donnez quelques caractéristiques de ce contrôleur, si non, expliquez pourquoi.
\end{enumerate}

Vous écrirez vos réponses dans ce même fichier.




\chapter{Le rapport}
Le rapport est sur 20 points.

\section{Processus}

\subsection{Rôle du fichier {\tt GNUmakefile} (1.5 points)}

\begin{itemize}
  \item lignes 1 à 87 : déclaration des variables globales nécessaires pour l'exécution des règles.
  \item règle \texttt{all} : exécute les règles \texttt{sources} et
    \texttt{tank.time} et compile les cibles
    \begin{itemize}
      \item \texttt{sujet.pdf}, \texttt{FD-2021-2022-M1-CC-sujet.tgz},
      \item \texttt{rapport-<user>.pdf}, \texttt{FD-2021-2022-M1-CC-rapport-<user>.tgz},
      \item \texttt{corrige.pdf} et \texttt{FD-2021-2022-M1-CC-corrige.tgz}.
    \end{itemize}
  \item règle \texttt{sources} : exécute \texttt{make all} pour les
    sous-répertoires \texttt{Alt} et \texttt{Spec}.
  \item règle \texttt{tank.time} 
      \begin{enumerate}
          \item Pour chaque controleur \texttt{Ctrl} et \texttt{CtrlVV}, est créé un fichier \texttt{tank.alt}. 
              Ce fichier est l'aggrégation de l'ensembles des fichiers AltaRica du répertoire \texttt{Alt}.
          \item On cacule ensuite un controleur a partir du fichier \texttt{tank.alt} et des specifications 
              \texttt{System.spe}.
          \item On itère ensuite dans le but de calculer un nouveau controleur et on le compare 
              avec l'ancien controlleur.
              Si les controleurs diffèrent on continue alors les itérations. 
              Sinon si il n'y a pas de différence on s'arrête.
          \item On reitère les operations ci dessus en incrémentant le nombres de pannes
              pour chaque controlleur tout en incrémentant le nombre de défaillance dans le système.
          \item Enfin on termine en exécutant la commande \texttt{make} dans tous les
              sous-répertoires : \texttt{Res} \texttt{Controleurs} \texttt{Graphs}
              \texttt{LaTeX}.
      \end{enumerate}
\end{itemize}

\subsection{Rôle de la constante {\tt nbFailures} et de l'assertion associée (0.5 point)}

La constante \texttt{nbFailures} corresponds au nombre de maximum de vannes pouvant tombées
en pannes simultanément (\texttt{Valve.stucked == 1}).
Cette condition est assurée par l'assertion :
\begin{verbatim}
# ligne 154 tank.alt
  nbFailures >= (V[0].stucked + V[1].stucked + V[2].stucked);
\end{verbatim}

\section{Résultats avec le contrôleur initial {\tt Ctrl}}
\subsection{Calcul d'un contrôleur}
\subsubsection{Avec 0 défaillance (0.5 point)}
\small{\lstinputlisting{Res/System0FCtrl.res}}
\lstinputlisting{Res/System0FCtrl0F1I.res}
\lstinputlisting{Res/System0FCtrl0F2I.res}

\paragraph{Interprétation des résultats}
Pour rappel les situations redoutés sont l'union des états critiques et des deadlock 
\textfff{SR := deadlock | NC;}
\begin{itemize}
    \item Avec 0 défaillance dans le systèmes on peut voir que le premier controlleur calculé
contient des états de situations redoutés.
    \item Après une projection le nouveau controlleur généré ne contient plus de situations redoutés.
    \item A la deuxième itération il n'y a plus de changement pour les controleurs obtenus,
les resultats sont stables le controleur est alors correct.
\end{itemize}

\subsubsection{Avec 1 défaillance (0.5 point)}
\small{\lstinputlisting{Res/System1FCtrl.res}}
\lstinputlisting{Res/System1FCtrl1F1I.res}
\lstinputlisting{Res/System1FCtrl1F2I.res}
\lstinputlisting{Res/System1FCtrl1F3I.res}

\paragraph{Interprétation des résultats}
\begin{itemize}
    \item Avec 1 défaillance le controleur initial contient 329 états redoutés.
    \item Après une première itération le controleur calculé à diminué considérablement les états critiques 
        et situations redoutés respectivement à 69 et 93.
        En revanche ce controlleur à introduit des deadlock dans le système...
    \item À la deuxième itération le système contient toujours autant d'etats critiques et de situations redoutés.
        et 3 deadlock en plus.
    \item A la troisième itération le système n'a pas évolué les resultats se sont stabilisé.
          Ce controlleur n'est pas correct, le controlleur n'est peut-être pas assez performant. 
\end{itemize}

\subsubsection{Avec 2 défaillances (0.5 point)}
\small{\lstinputlisting{Res/System2FCtrl.res}}
\lstinputlisting{Res/System2FCtrl2F1I.res}
\lstinputlisting{Res/System2FCtrl2F2I.res}
\lstinputlisting{Res/System2FCtrl2F3I.res}
\lstinputlisting{Res/System2FCtrl2F4I.res}

\paragraph{Interprétation des résultats}
\begin{itemize}
    \item La controleur initial contient 551 états redoutés.
    \item En itérant le processus de calcul l'etat se stabilise à la 4ème itération, cependant
        le controleur n'est pas correct il contient 107 états critiques, 270 situations redoutés.
\end{itemize}

\subsubsection{Avec 3 défaillances (0.5 point)}
\small{\lstinputlisting{Res/System3FCtrl.res}}
\lstinputlisting{Res/System3FCtrl3F1I.res}
\lstinputlisting{Res/System3FCtrl3F2I.res}
\lstinputlisting{Res/System3FCtrl3F3I.res}

\paragraph{Interprétation des résultats}
\begin{itemize}
    \item La controleur initial contient 617 situations redoutés.
    \item En itérant le processus de calcul l'etat se stabilise à la 3ème itération, cependant
        le controleur n'est pas correct il contient 27 situations redoutés.
\end{itemize}

\subsection{Bilan avec le contrôleur initial (1 point)}
% A COMPLETER

\section{Construction d'un contrôleur initial plus performant}
\subsection{Rôle du composant {\tt ValveVirtual}(2 points)}
\includegraphics[height=.2\textheight,width=.5\textwidth]{Graphs/Valve-modes.pdf}
\includegraphics[height=.2\textheight,width=.5\textwidth]{Graphs/ValveVirtual-modes.pdf}
% A COMPLETER en expliquant sa sémantique et son rôle.

\subsection{Rôle du composant {\tt CtrlVV} (4 points)}
% A COMPLETER en expliquant les mécanismes mis en oeuvre, leurs rôles et les
% avantages de ce contrôleur par rapport au précédent CtrlVV.

\section{Résultats avec le contrôleur {\tt CtrlVV}}
\subsection{Calcul d'un contrôleur}
\subsubsection{Avec 0 défaillance (0.5 point)}
\small{\lstinputlisting{Res/System0FCtrlVV.res}}
\lstinputlisting{Res/System0FCtrlVV0F1I.res}
\lstinputlisting{Res/System0FCtrlVV0F2I.res}

\paragraph{Interprétation des résultats}
% A COMPLETER

\subsubsection{Avec 1 défaillance (0.5 point)}
\small{\lstinputlisting{Res/System1FCtrlVV.res}}
\lstinputlisting{Res/System1FCtrlVV1F1I.res}
\lstinputlisting{Res/System1FCtrlVV1F2I.res}
\lstinputlisting{Res/System1FCtrlVV1F3I.res}
\lstinputlisting{Res/System1FCtrlVV1F4I.res}

\paragraph{Interprétation des résultats}
% A COMPLETER

\subsubsection{Avec 2 défaillances (0.5 point)}
\small{\lstinputlisting{Res/System2FCtrlVV.res}}
\lstinputlisting{Res/System2FCtrlVV2F1I.res}
\lstinputlisting{Res/System2FCtrlVV2F2I.res}
\lstinputlisting{Res/System2FCtrlVV2F3I.res}

\paragraph{Interprétation des résultats}
% A COMPLETER

\subsubsection{Avec 3 défaillances (0.5 point)}
\small{\lstinputlisting{Res/System3FCtrlVV.res}}
\lstinputlisting{Res/System3FCtrlVV3F1I.res}
\lstinputlisting{Res/System3FCtrlVV3F2I.res}
\lstinputlisting{Res/System3FCtrlVV3F3I.res}

\paragraph{Interprétation des résultats}
% A COMPLETER

\subsection{Bilan avec le contrôleur CtrlVV (1 point)}
% A COMPLETER

\section{Une première optimisation des contrôleurs pour améliorer le débit aval}
\subsection{Une optimisation basée sur les priorités (1 point)}
\small{\lstinputlisting{ControleursOpt/Optimisation.alt}}
% A COMPLETER en expliquant en quoi consiste l'optimisation mise en place.

\subsection{Calcul des contrôleurs optimisés avec {\tt CtrlVV}}
\paragraph{Avec 0 défaillance}\ \\
box{\small{\lstinputlisting{Res/System0FCtrlVV0F2I_opt.res}}}


\paragraph{Avec 1 défaillance}\ \\
box{\small{\lstinputlisting{Res/System1FCtrlVV1F4I_opt.res}}}


\paragraph{Avec 2 défaillances}\ \\
box{\small{\lstinputlisting{Res/System2FCtrlVV2F3I_opt.res}}}


\paragraph{Avec 3 défaillances}\ \\
box{\small{\lstinputlisting{Res/System3FCtrlVV3F3I_opt.res}}}


\subsection{Bilan avec la première optimisation du contrôleur CtrlVV (1 point)}
% A COMPLETER

\section{Une deuxième optimisation (3 points)}
Il est possible d'obtenir de meilleurs résultats que les précédents par au moins deux façons.
\begin{enumerate}
\item En utilisant un meilleur ordre pour les priorités entre événements.
\item En introduisant cet objectif dans le système de calcul de point fixe des actions du contrôleur.
\end{enumerate}

Vous devez proposer une des deux optimisations conduisant à une solution dans laquelle le débit de la vanne aval est le moins souvent possible décrémenter, voire jamais. Pour cela, vous pouvez~:
\begin{itemize}
\item Meilleur ordre sur les événements.
  \begin{enumerate}
  \item Modifier le fichier \texttt{ControleursOpt/Optimisation.alt}.
  \item Faites \texttt{make}.
  \item Quand vos résultats sont satisfaisants, notez les, puis copiez votre fichier dans \texttt{ControleursOpt/Optimisation-2.alt}.
  \item Remettez le fichier \texttt{ControleursOpt/Optimisation.alt} d'origine.
  \item Faites \texttt{make}.
  \end{enumerate}
\item Meilleur système d'équations au point fixe.
  \begin{enumerate}
  \item Modifier le fichier \texttt{Spec/System.spe}.
  \item Faites \texttt{make}.
  \item Quand vos résultats sont satisfaisants, notez les, puis copiez votre fichier dans \texttt{Spec/System-2.spe}.
  \item Remettez le fichier \texttt{Spec/System.spe} d'origine.
  \item Faites \texttt{make}.
  \end{enumerate}
\end{itemize}

\paragraph{Le nouvel ordre}\ \\
\small{\lstinputlisting{ControleursOpt/Optimisation-2.alt}}
% A COMPLETER en décrivant les résultats obtenus

\paragraph{Le nouveau système d'équations}\ \\
\small{\lstinputlisting[texcl]{Spec/System-2.spe}}
% A COMPLETER en décrivant les résultats obtenus

\section{Conclusion sur la synthèse de contrôleur (1 point)}
% A COMPLETER

\end{document}
